\documentclass[a4paper]{article}

\usepackage{fontspec}
\usepackage{mathpazo}
\setmainfont
     [ BoldFont       = texgyrepagella-bold.otf ,
       ItalicFont     = texgyrepagella-italic.otf ,
       BoldItalicFont = texgyrepagella-bolditalic.otf ]
     {texgyrepagella-regular.otf}
\setmainfont{Gill Sans MT}

\usepackage[english]{babel}
\usepackage[utf8]{inputenc}
\usepackage{amsmath}
\usepackage{graphicx}
\usepackage[colorinlistoftodos]{todonotes}
\usepackage{physics}

%\DeclareMathOperator{\Tr}{Tr}

\title{Homework 3 : Aircraft Collision Avoidance Analyses using Reachability}

\author{David McPherson}

\date{\today}

\begin{document}
\maketitle

\section{Linear Velocity Control}
Consider a system of two planes (Dubin's vehicles) on a collision course.
Our two planes can only control their linear velocity and cannot escape into a veering mode.
Our system with no turning and using relative coordinates only is:

\begin{align*}
     \dot{x}_r    &= -u + d cos(\psi_r)
  \\ \dot{y}_r    &=  d sin(\psi_r)
  \\ \dot{\psi}_r &=  0
\end{align*}

\section{Mode Switching Control}

\end{document}
